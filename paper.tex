\documentclass[12pt]{article}

\title{Using Recombination Markov Chains to Analyze the Relative Effects of 2025 Redistricting Policies}
\date{December 2025}
\author{Caden Lee and Sydney Hoang}

\begin{document}
  \maketitle
  \begin{abstract}
    Redistricting is the process of redrawing electoral district boundaries based on regional population. Novel redistricting policies are a major source of controversy in the 2025 United States political climate. Recent computer-assisted analysis involves the stochastic construction of various redistricting schemes using methods such as Recombination Markov Chains (Recom). With Recom, any redistricting proposal is normed against the constructed schemes in order to detect gerrymandering. Interpretation of these results is often supplemented by a metric, such as a Polsby-Popper or Reock score, suggesting the geographical compactness of a region. However, as these scores solely prioritize geometric compactness, they are effectively blind to gerrymandering of smaller localities, such as counties or cities.

    Our study applies a mathematical model to analyze the current district maps resulting from the 2025 redistricting policies. Using publicly available data, we implemented the Recombination Markov Chain algorithm in Python, modeling the probability distribution of different demographic groups. Our Recom model was utilized in conjunction with a locality preservation model of compactness to provide additional insights into the relative effects of recent redistricting proposals. Preliminary results strongly indicated the presence of gerrymandering undetected by geometric compactness scores, suggesting a significant impact on future voting processes.
  \end{abstract} 

  \section{Literature Review}

Read up on this stuff

  \include{sections/methodology}
  
  \include{sections/acknowledgements}


\end{document}


